\documentclass[preprint,12pt,authoryear]{elsarticle}
\usepackage[T1]{fontenc}
\usepackage[utf8]{inputenc}
\usepackage{amsmath,amssymb,amsthm}
\usepackage{geometry}
\geometry{a4paper, margin=2.5cm}

% Theorem environments
\newtheorem{theorem}{Theorem}
\newtheorem{proposition}[theorem]{Proposition}

% Short, EL-style letter
\begin{document}
\begin{frontmatter}

\title{Signal Amplification in Market Manipulation Detection}

\author[qub]{Yongsheng Dai}
\ead{ydai09@qub.ac.uk}
\author[uu]{Barry Quinn}
\ead{b.quinn1@ulster.ac.uk}
\author[qbs]{Fearghal Kearney}
\ead{f.kearney@qub.ac.uk}

\address[qub]{School of EEECS, Queen's University Belfast}
\address[uu]{Ulster University Business School, Ulster University}
\address[qbs]{Queen's Business School, Queen's University Belfast}

\begin{abstract}
We study a manipulator--detector game and show that a composite detection signal strictly outperforms individual features when manipulation strategies are complementary. Under weak regularity and $\operatorname{Cov}(r,c)>0$, the optimal linear detector $S=w_1 r + w_2 c + \varepsilon$ achieves a strictly higher detection objective than either feature alone, and induces strategic deterrence by reducing equilibrium manipulation intensity. A compact numerical illustration calibrated to mature-market conditions confirms meaningful amplification and interior equilibria. The result provides a simple design principle for surveillance: combine domain-specific features when they co-move under manipulation.
\end{abstract}

\begin{keyword}
Market manipulation \sep detection \sep microstructure \sep game theory
\JEL G14 \sep G18 \sep C72
\end{keyword}

\end{frontmatter}

% --- Main text ---

\section{Setup}
The manipulator chooses $(r,c)\in\mathbb{R}_+\times[0,1]$ (rush-order and cancellation intensity). The detector observes a noisy linear signal
\begin{equation}
S=w_1 r + w_2 c + \varepsilon,\qquad \varepsilon\sim \mathcal{N}(0,\sigma^2),\label{eq:signal}
\end{equation}
and triggers detection when $S>\tau$. Let the manipulation benefit be $Q(\delta r + \gamma c)$ and the private cost $\tfrac{1}{2}(k r^2 + \ell c^2)$; a penalty $L>0$ is imposed upon detection. We analyze detection performance using a standard linear objective (e.g., power at fixed size or a weighted Type~I/II loss), which in this Gaussian setting is monotonically increasing in a signal-to-noise ratio (SNR) measure of $S$.

\section{Main Result}
Let $\mu_r,\mu_c>0$ and variances $\sigma_r^2,\sigma_c^2$ characterize $(r,c)$ under manipulation, with $\operatorname{Cov}(r,c)=\delta>0$, and assume $\varepsilon$ is independent of $(r,c)$.

\begin{theorem}[Signal Amplification]\label{thm:amplification}
For the linear detector $S=w_1 r + w_2 c + \varepsilon$, the optimal weights $(w_1^*,w_2^*)$ strictly dominate any single-feature detector whenever $\operatorname{Cov}(r,c)>0$. In particular,
\begin{equation}
\operatorname{SNR}(w_1^*,w_2^*)\;>\;\max\big\{\operatorname{SNR}(1,0),\;\operatorname{SNR}(0,1)\big\}+\varepsilon(\delta),\quad \varepsilon(\delta)>0,\label{eq:snr-dominance}
\end{equation}
with $\varepsilon(\delta)$ strictly increasing in $\delta$.
\end{theorem}

\begin{proof}[Proof sketch]
Write the (scaled) detection objective as the Rayleigh quotient $\tfrac{w^\top \Sigma w}{w^\top w + \sigma^2}$, where $w=(w_1,w_2)$ and $\Sigma$ is the second-moment matrix of $(r,c)$ under manipulation, including cross term $\operatorname{Cov}(r,c)=\delta>0$. The maximizer uses both coordinates unless the off-diagonal is zero; restricting to axes $(1,0)$ or $(0,1)$ is suboptimal when $\delta>0$ because the cross term raises the quadratic form strictly above the best coordinate direction. Monotonicity of common detection criteria in this ratio yields \eqref{eq:snr-dominance}.
\end{proof}

\begin{proposition}[Strategic deterrence]\label{prop:deterrence}
In an interior equilibrium, the optimal manipulation intensities satisfy $\partial r^*/\partial w_1<0$ and $\partial c^*/\partial w_2<0$.
\end{proposition}

\noindent Sketch: First-order conditions equate marginal benefits to detection-adjusted marginal costs. A higher weight $w_1$ (or $w_2$) raises detection probability for a given $(r,c)$, shifting the FOCs inward; the implicit-function theorem yields the stated signs under standard regularity and concavity.

\section{Illustration}
For the calibration $Q=100,\; (\delta,\gamma)=(1.0,0.8),\; (k,\ell)=(50,40),\; L=100,\; \sigma=0.5$, numerical optimization yields an interior equilibrium and a strictly higher detection objective for the composite detector. The precise magnitude (e.g., an \emph{illustrative} 80\% improvement over single features in one configuration) depends on covariance and noise and is robust in a neighborhood around the calibration. Replication scripts are provided in the accompanying repository.

\section{Implications}
Theorem~\ref{thm:amplification} provides a simple design rule: when manipulation features co-move during abuse, linear combination improves detection and deters manipulation in equilibrium (Proposition~\ref{prop:deterrence}). Empirically, researchers can test whether composite AUC exceeds the best single-feature AUC and whether surveillance upgrades reduce manipulation intensity.

\section{Conclusion}
Composite detection strictly dominates single-feature monitoring under complementary manipulation strategies and induces deterrence. This letter isolates the core condition and delivers a tractable guideline for market surveillance system design.

\bibliographystyle{elsarticle-harv}
\bibliography{references}

\end{document}

